
\documentclass[11pt,a4paper]{abntex2}
%\usepackage[latin1]{inputenc}
\usepackage[utf8]{inputenc}
\usepackage[brazil]{babel}
\usepackage[T1]{fontenc}
\usepackage{amsmath}
\usepackage{makeidx}
\usepackage{lipsum}
\usepackage{graphicx}
\usepackage{cite} 
\usepackage{indentfirst}
\usepackage{listings}
\usepackage{float}
\usepackage{multirow}
\lstset{
	basicstyle = {\footnotesize},
	numbers=left,
	firstline=1,
	language=Java,
	%frame=leftline|topline,
	frame=single,
	tabsize=2,
	breaklines=true
}



%\lstset{
%language = C++, % Linguagem de programação
%basicstyle = \footnotesize, % Tamanho da fonte do código
%numbers = left, % Posição da numeração das linhas
%numberstyle = \tiny\color{blue}, % Estilo da numeração de linhas
%stepnumber = 1, % Numeração das linhas ocorre a cada quantas linhas?
%numbersep = 10pt, % Distância entre a numeração das linhas e o código
%backgroundcolor = \color{white}, % Cor de fundo
%showspaces = false, % Exibe espaços com um sublinhado
%showstringspaces = false, % Sublinha espaços em Strings
%showtabs = false, % Exibe tabulação com um sublinhado
%frame = trBL, % Envolve o código com uma moldura, pode ser single ou trBL
%rulecolor = \color{black}, % Cor da moldura
%tabsize = 2, % Configura tabulação em x espaços
%captionpos = b, % Posição do título pode ser t (top) ou b (bottom)
%breaklines = true, % Configura quebra de linha automática
%breakatwhitespace= false, % Configura quebra de linha
%title = \lstname, % Exibe o nome do arquivo incluido
%caption = \lstname, % Também é possível usar caption no lugar de title
%keywordstyle = \color{blue}, % Estilo das palavras chaves
%commentstyle = \color{dkgreen}, % Estilo dos Comentários
%stringstyle = \color{mauve}, % Estilo de Strings
%escapeinside = {\%*}{*)}, % Permite adicionar comandos LaTeX dentro do seu código
%morekeywords     ={*,...} % Se quiser adicionar mais palavras-chave
%}%

\begin{document}
	\begin{titlepage}
		\vfill
		\begin{center}
			{\large \textbf{Universidade Federal de Ouro Preto}} \\[2.5cm]
			
			  
			{\large \textbf{Mateus Oliveira dos Santos - 11.2.8093}}\\[4cm]
			
			
			{\Large Trabalho Extra}\\[4cm]
			
			\hspace{.45\textwidth} %posiciona a minipage
			\begin{minipage}{.5\textwidth}
				\large Trabalho Extra apresentado a disciplina de Avaliação e Desempenho de Sistemas Computacionais do curso de Engenharia de Computação do  Instituto de Ciências Exatas e Aplicadas da  Universidade  Federal de Ouro Preto.\\[1cm]
				Professor Alexandre Magno de Sousa.
			\end{minipage}
			\vfill
			
			\vspace{2cm}
			
			\large \textbf{João Monlevade - MG}
			
			\large \textbf{\date}
		\end{center}
	\end{titlepage}	
	
\section*{\textbf{Introdução}}

O trabalho extra proposto tem como objetivo consolidar os conhecimentos adquiridos em sala de aula e colocar em prática a análise de modelos de sistemas multiclasse. 

Apresentamos os enunciado proposto, os softwares e a metodologia utilizada, a definição do modelo multiclasse, e logo mais a extração dos dados do problema seguidos das resoluções dos questionamentos do enunciado, seguindo a metodologia proposta. 

\section*{\textbf{Enunciado}}
	
	A database server has one CPU and one disk. The server’s workload is composed of trivial queries	that arrive at a rate of 10 tps, complex queries that arrive at a rate of 0.1 tps, and of a batch 	workload that generates a report. When the report generation completes, a new report generation is
	started in 15 minutes. Table 1 provides workload related information. Each physical I/O demands	0.015 msec of CPU time and 9 msec of disk service time. The last row of Table 1 indicates how much	CPU time is required by transactions of each workload in addition to the CPU time related to I/Os.
		
	\begin{itemize}
		\item What kind of QN should be used to model this situation? Specify the type of workload, resources and network.
		\item Find the average response time and the average throughput for each of the three workloads.
		\item Find the utilization of the CPU and of the disk.
		\item Finds the residence times at the CPU and at the disk for each of the three workloads.
		
	\end{itemize}
	
	\begin{table}[htbp]
		\centering
		\caption{Data for exercise} 
		
		\begin{tabular}{lccc}
			\toprule
		 Workload    & Trivial    &Complex    & Report\\
			\midrule
		Avg. Number of SQL Calls         & 3.5  & 20.0  & 120.0    \\
		Avg. Number of I/Os per SQL Call & 5.0  & 15.0 & 40.0   \\
		DB Buffer Hit Ratio (in \%)      & 70.0 & 80.0   & 30.0     \\
		Non I/O Related CPU Time (msec)	 & 30.0 & 180.0   & 1250.0    \\
		%\multicolumn{7}{l}{\textit{caption fonte wwww}} \\
			\bottomrule
		\end{tabular}%
		%\namedlegend{teste legend}
		\label{tab:addlabel}%
	\end{table}%
	
\newpage

\section*{\textbf{Software Utilizados}}
	Segue a lista dos software utilizados para desenvolvimento do trabalho.
	
	\begin{itemize}
	\item  Java Modelling Tools - JMT disponível em \textit{http://jmt.sourceforge.net/Requirements.html}
	\item IDE Latex TexStudio
	\end{itemize}



\section*{\textbf{Metodologia de Resolução}}
	A metologia utilizada para resolver o problema proposto de modelo multiclasse em questão foi a sugerida na prova de Avaliação e Desempenho de Sistemas pelo Professore Alexandre Magno de Sousa. 


	
	\begin{enumerate}
	\item  Resolver os submodelos isoladamente, porém analisando o impacto que um infere ou outro;
	\item  obter as utilizações para o submodelo aberto para cada classe de carga;
	\item encontrar as utilizações para cada recurso do submodelo aberto(soma das utilizações encontradas);
	\item encontrar as demandas estendidas para o submodelo fechado;
	\item utilizar algoritmo de MVA para encontrar o throughput, tempo de respora e tamanho da fila para o modelo fechado;
	\item calcular o tempo de residência para cada classe de transação do submodelo aberto;
	\item calcular o tempo de resposta para cada classe de transação.  
	\end{enumerate}
	

\section*{\textbf{Definição do Modelo} }

Estação seção dedica-se a resolução passo-a-passo do problemas proposto.
		\begin{itemize}
			\item \textbf{TipoQN:} Mista
			\item \textbf{Recursos:} 1 CPU e 1 Disco
			\item \textbf{Carga:} Heterogênia
			\item \textbf{Rede:} Aberta e Fechada
		\end{itemize}
% % % % % % % % % % %  falta figura % % % % % % % % % % % % % % % %		
	
	
	
		
\section*{\textbf{Extração de dados do problema}}	

Devemos seguir a metodologia para encontrar o tempo médio de resposta, o tempo de residêcia, as utilizações e througput solicitados no enunciado.

Temos que $S_{cpu}=0.000015$ e $S_{disco}=0.009$.



Dados inferidos com base no contexto do problema proposto. Os subscritos \textit{t}, \textit{c} e \textit{r}, respectivamente são \textit{trivial}, \textit{complex} e \textit{report}. O $\delta$ é o tipo de transação e o $\psi$ quando usado representa o recurso. Os sobrescritos \textit{IO}, \textit{Hit}, \textit{B}, \textit{T} e \textit{e} respectivamente são referência a acesso a disco, referência acessos no \textit{buffer}, carga \textit{batch}, carga de transação e demanda estendida.

\begin{table}[htbp]
\centering
\caption{Dados problema} 
\begin{tabular}{lccc}
	\toprule
	Equações    & Trivial    & Complex    & Report\\
	\midrule
	$\lambda_{\delta}$       & $\lambda_{t} = 10 tps$ & $\lambda_{c} = 0.1 tps$  & ?\\[3pt]
	$V_{(cpu,\delta)}$ & $V_{(cpu,t)}=?$  & $V_{(cpu,c)}=?$ & $V_{(cpu,r)}=?$   \\[3pt]
	$V_{(disco,\delta)}=SqlCall*\frac{IO}{SqlCall}*P^{IO}_{\delta}$ & $V_{(disco,t)}=5.25$ & $V_{(disco,c)}=60.0$   & $V_{(disco,r)}=3360.0$ \\[3pt]
	$P^{Hit}_{\delta}$ & $P^{Hit}_{t}=0.7$ & $P^{Hit}_{c}=0.8$ & $P^{Hit}_{r}=0.3$ \\[3pt]
	$P^{IO}_{\delta}$  & $P^{IO}_{t}=0.3$  & $P^{IO}_{c}=0.2$  & $P^{IO}_{r}=0.7$   \\[3pt]
	%\multicolumn{7}{l}{\textit{caption fonte wwww}} \\
	\bottomrule
\end{tabular}%
%\namedlegend{teste legend}
\label{tab:addlabel}%
\end{table}%

\section*{\textbf{Resolvendo o submodelo aberto}}

Cálculo da demanda de CPU:

\begin{equation}
D^{IO}_{(cpu,\delta)} =V_{(cpu,\delta)} * S_{cpu} *P^{IO}_{\delta}
\end{equation}

\begin{equation}
D_{(cpu,\delta)} = D^{IO}_{(cpu,\delta)} + D^{Hit}_{(cpu,\delta)}
\end{equation}

O $D^{Hit}_{(cpu,\delta)}$ é dado pela última linha da tabela 1.

 
\begin{table}[htbp]
	\centering
	\caption{Demandas de CPU por transação} 
	\begin{tabular}{lccc}
	\toprule
	Carga & $D^{Hit}_{(cpu,\delta)}$ & $D^{IO}_{(cpu,\delta)}$ & $D_{(cpu,\delta)}$\\
		\midrule
	    Trivial (s)& 0.030 & $0.07875*10^{-3}$ & $30.07875*10^{-3}$\\[3pt]
		Complex (s)& 0.180 & $0.9*10^{-3}$     & $180.9*10^{-3}$   \\[3pt]
		Report (s) & 1.250 & $50.4*10^{-3}$    & $130.04*10^{-3}$ \\[3pt]
		%\multicolumn{7}{l}{\textit{caption fonte wwww}} \\
		\bottomrule
	\end{tabular}%
	%\namedlegend{teste legend}
	\label{tab:addlabel}%
\end{table}%

Cálculo da demanda de Disco:

\begin{equation}
D_{(disco,\delta)} =V_{(disco,\delta)} * S_{disco}
\end{equation}

\begin{table}[htbp]
	\centering
	\caption{Demandas de Disco por transação} 
	\begin{tabular}{lc}
		\toprule
		Carga      & $D_{(dsico,\delta)}$\\
		\midrule
		Trivial (s)& 0.04725 \\[3pt]
		Complex (s)& 0.540 \\[3pt]
		Report (s) & 30.24 \\[3pt]
		%\multicolumn{7}{l}{\textit{caption fonte wwww}} \\
		\bottomrule
	\end{tabular}%
	%\namedlegend{teste legend}
	\label{tab:addlabel}%
\end{table}%


Cálculo para utilização do disco e da CPU para o submodelo aberto

\begin{equation}
U_{(\psi,\delta)} =D_{(\psi,\delta)} * \lambda_{\delta}
\end{equation}

\begin{equation}
U^{T}_{\psi} = \sum_{\delta=0}^{K} U_{(\psi,\delta)}
\end{equation}


\begin{table}[htbp]
	\centering
	\caption{Utilização do submodelo aberto} 
	\begin{tabular}{lccc}
		\toprule
		Carga      & $U_{(disco,\delta)}$& $U_{(cpu,\delta)}$ \\
		\midrule
		Trivial (s)            & 0.4725 & 0.3007875\\[3pt]
		Complex (s)            & 0.054  & 0.01809\\[3pt]
		$U^{T}_{\psi}$ & 0.5265 & 0.32609 \\[3pt]
		%\multicolumn{7}{l}{\textit{caption fonte wwww}} \\
		\bottomrule
	\end{tabular}%
	%\namedlegend{teste legend}
	\label{tab:addlabel}%
\end{table}%

Temos que o tamanho total da fila por recurso é dado por,

\begin{equation}
N^{T}_{\psi} = \frac{U^{T}_{\psi}}{1 - U^{T}_{\psi}}
\end{equation}

Logo podemos calcular o tamanho da fila para cada tipo de transação como um percente do tamanho total. Portanto,

\begin{equation}
N^{T}_{(\psi,\delta)} = N^{T}_{\psi}*\frac{U^{T}_{(\psi,\delta)}}{U^{T}_{\psi}}
\end{equation}

\begin{table}[htbp]
	\centering
	\caption{Tamanho da Fila CPU e Disco - Complex e Trivial} 
	\begin{tabular}{lccc}
		\toprule
		Carga            & CPU    & Disco \\
		\midrule
		$N^{T}_{(\psi,t)}$ & 0.44627 & 0.99788\\[3pt]
		$N^{T}_{(\psi,c)}$ & 0.02684 & 0.11404\\[3pt]
		$N^{T}_{\psi}$   & 0.48387 & 1.11193 \\[3pt]
		%\multicolumn{7}{l}{\textit{caption fonte wwww}} \\
		\bottomrule
	\end{tabular}%
	%\namedlegend{teste legend}
	\label{tab:addlabel}%
\end{table}%



Agora já podemos calcular o tempo de residência de cada carga do sistema aberto pois,

\begin{equation}
R^{T}_{(\psi,\delta)} = \frac{N^{T}_{(\psi,\delta)}}{\lambda_{\delta}}
\end{equation}

Logo,

\begin{equation}
R^{T}_{\delta} = \sum_{\psi=0}^{K} R^{T}_{(\psi,\delta)}
\end{equation}

\begin{table}[htbp]
	\centering
	\caption{Tempo de Residência submodelo aberto - CPU e Disco} 
	\begin{tabular}{lccc}
		\toprule
		Carga              & CPU     & Disco     & $R^{T}_{\delta}$ \\
		\midrule
		$R^{T}_{(\psi,t)}$ & 0.044627 & 0.099788 & 0.14442  \\[3pt]
		$R^{T}_{(\psi,c)}$ & 0.2684   & 1.1404   & 1.4088 \\[3pt]
		   
		%\multicolumn{7}{l}{\textit{caption fonte wwww}} \\
		\bottomrule
	\end{tabular}%
	%\namedlegend{teste legend}
	\label{tab:TR_t}%
\end{table}%

Recalculando os valores ao final da resolução do submodelo aberto temos:

\begin{equation}
V{(\psi,\delta)} = \frac{D_{(\psi,\delta)}}{S_{\psi}}
\end{equation}


\begin{table}[htbp]
\centering
\caption{Dados do Problema Recalculado} 
\begin{tabular}{lccc}
	\toprule
	Equações    & Trivial    & Complex    & Report\\
	\midrule
	$\lambda_{\delta}$       & $\lambda_{t} = 10 tps$ & $\lambda_{c} = 0.1 tps$  & ?\\[3pt]
	$V_{(cpu,\delta)}$ & $V_{(cpu,t)}=2005$  & $V_{(cpu,c)}=12060$ & $V_{(cpu,r)}=86693.33$   \\[3pt]
	$V_{(disco,\delta)}=SqlCall*\frac{IO}{SqlCall}*P^{IO}_{\delta}$ & $V_{(disco,t)}=5.25$ & $V_{(disco,c)}=60.0$   & $V_{(disco,r)}=3360.0$ \\[3pt]
	$P^{Hit}_{\delta}$ & $P^{Hit}_{t}=0.7$ & $P^{Hit}_{c}=0.8$ & $P^{Hit}_{r}=0.3$ \\[3pt]
	$P^{IO}_{\delta}$  & $P^{IO}_{t}=0.3$  & $P^{IO}_{c}=0.2$  & $P^{IO}_{r}=0.7$   \\[3pt]
	%\multicolumn{7}{l}{\textit{caption fonte wwww}} \\
	\bottomrule
\end{tabular}%
\end{table}

\section*{\textbf{Resolvendo o submodelo \textit{batch}}}

Para calcular a demanda do modelo \textit{batch} temos que calcular a demanda estendida do modelo fechado. Este calculo é necessário pois modelo de transação impacta no modelo fechado, está é uma abordagem para resolver modelos multiclasse.

\begin{equation}
	D^{e}_{(\psi,batch)} = \frac{D_{(\psi,batch)}}{1 - U^{T}_{\psi}}
\end{equation}

\begin{equation}
	D^{e}_{(cpu,r)} = 1.92963
\end{equation}

\begin{equation}
	D^{e}_{(dsico,r)} = 63.8664
\end{equation}
 
Agora podemos utilizaremos o módulo Java Modelling Tools - JMT que utiliza o algoritmo MVA para encontrar o tamanho da fila e o throughput do sistema fechado. Como exemplo a diferenção do tempo de resposta analisado para a carde de transação \textit{complex} com influência do modelo \textit{batch} é aproximandamente 55\% maior que desconsiderando o impacto. 


\begin{table}[htbp]
	\centering
	\caption{Tamanho da Fila -  Report} 
	\begin{tabular}{lccc}
		\toprule
		Carga       & CPU    & Disco \\
		\midrule
		$N^{B}_{\psi}$ & 0.02902 & 0.97097\\[3pt]
		$X^{B}_{cpu}$ & 0.02    &   \\[3pt]
		%\multicolumn{7}{l}{\textit{caption fonte wwww}} \\
		\bottomrule
	\end{tabular}%
	%\namedlegend{teste legend}
	\label{tab:addlabel}%
\end{table}%

Com estamos resolvendo o sistema fechado isoladamente ele não experimenta fila logo, uma solução por MVA  temos,

\begin{equation}
R_{\psi} = V_{\psi}*S_{\psi} = D_{\psi}
\end{equation}

Assim temos que o tempo de residência para carga \textit{batch},

\begin{table}[htbp]
	\centering
	\caption{Tempo de Residência -  Report} 
	\begin{tabular}{lccc}
		\toprule
		Carga       & CPU    & Disco & Total \\
		\midrule
		$R^{B}_{\psi}$ & 1.92963 & 63.8664 & 65.79603 \\[3pt]
		%\multicolumn{7}{l}{\textit{caption fonte wwww}} \\
		\bottomrule
	\end{tabular}%
	%\namedlegend{teste legend}
	\label{tab:TR_b}%
\end{table}%




\section*{\textbf{Resolvendo submodelo aberto com influência do submodelo fechado}}

Para encontrar o tempo de resposta do submodelo aberto devemos considera também o impacto do submodelo fechado.
Portanto temos que o tempo de residência é dado por,

\begin{equation}
R^{'}_{(\psi,\delta)} = D_{(\psi,\delta)}*(1 + N_{(\psi,batch)})*\frac{1}{(1 - U_{(\psi,\delta)})}
\end{equation} 

\begin{equation}
R_{\delta} = \sum_{\psi =1}^{K} R^{'}_{(\psi,\delta)} 
\end{equation} 

onde $\delta \in$ \textit{\textbf{transações}}.



\begin{table}[htbp]
	\centering
	\caption{Tempo Médio de Resposta submodelo aberto com impacto batch} 
	\begin{tabular}{lccc}
		\toprule
		Carga                    & Trivial  & Complex \\
		\midrule
		$R^{'}_{(cpu,\delta)}$   &  0.04592 & 0.27402 \\[3pt]
		$R^{'}_{(dsico,\delta)}$ &  0.19668 & 2.24778\\[3pt]
		$R_{\delta}$             &  0.2426  & 2.5218 \\[3pt]
		%\multicolumn{7}{l}{\textit{caption fonte wwww}} \\
		\bottomrule
	\end{tabular}%
	%\namedlegend{teste legend}
	\label{tab:TMR_t}%
\end{table}%

Como o tempo de resposta mudou o número de tarefas no recurso também se altera. Temos que,

\begin{equation}
N^{T}_{\psi} = \sum_{\delta}^{T} \lambda_{\delta}*R_{(\psi,\delta)}
\end{equation}


\begin{table}[htbp]
	\centering
	\caption{Tamanho da Fila -  Transações} 
	\begin{tabular}{lcc}
		\toprule
		Carga          & CPU    & Disco \\
		\midrule
		$N^{T}_{\psi}$ & 0.48660 & 2.19157\\[3pt]
		%\multicolumn{7}{l}{\textit{caption fonte wwww}} \\
		\bottomrule
	\end{tabular}%
	%\namedlegend{teste legend}
	\label{tab:addlabel}%
\end{table}%


\section*{\textbf{Tempo de Resposta, Tempo de Residncia }}

 O tempo médio de resposta para o modelo aberto se encontra na tabela [\ref{tab:TMR_t}] e para o modelo fechado na tabela [\ref{tab:TR_b}] que é o mesmo que o tempo de residência. O tempo de residência para o modelo fechado segue na tabela [\ref{tab:TR_t}].

O throughput para carga batch é defino como,

\begin{equation}
X_{0} = \frac{N}{R + Z}
\end{equation}

Como $N=1$ pois só temos um serviço de gerar relatório em execução. O tempo de pensar é dado no problema como 15 minutos logo $Z = 15*60 = 900 s$. O tempo médio de resposta já está calculo na tabela [\ref{tab:trR}], $R=65.79603$. Logo $X_{batch}= 0.0010354 tps$, já dados $X_{t}=10.0 tps$ e $X_{c}=0.1 tps$ . 

\section*{\textbf{Utilização}}

Temos que,

\begin{equation}
U_{\psi}  = U^{T}_{\psi} + U^{batch}_{\psi}
\end{equation}

Dado que $U^{T}_{cpu}=0.32609$  e $U^{T}_{disco}=0.5265$ e

\begin{equation}
U_{\psi}  = X_{0}* D_{\psi}
\end{equation}

temos que,
\begin{equation}
U^{batch}_{cpu} = X_{batch}* D^{e}_{(cpu,batch)} = 0.0010354 *1.92963=0.00199 
\end{equation}

\begin{equation}
   U^{batch}_{disco} = X_{batch}* D^{e}_{(disco,batch)} = 0.0010354*63.8664=0.06612 
\end{equation}

Logo,$U_{cpu} \approx 0.32808 \approx 32.80\% $ e  $U_{disco} \approx 0.59262 \approx 59.26\% $.


\section*{\textbf{Conclusão}}

Por meio do trabalho proposto, podemos reafirmar que em uma análise de modelo multiclasse o impacto que os submodelos geram um no outro não pode ser ignorado, haja vista que o tempo de respota para a classe aberta da carga \textit{complex} \´{e} aproximadamente 55\% maior que o tempo de resposta sem considerar o impacto do modelo fechado.

	%\begin{figure}[H]
%		\centering%
%		\includegraphics[scale=0.50]{geral.png}
%		\caption{Diagrama Projeto TP1}
%		\label{fig:geraltp1}
%	\end{figure}

\section*{\textbf{Bibliografia}}
JAIN, Raj. The Art of Computer System
Performance Analysis. EUA: John Wiley &
Sons, 1991.

	
\end{document}
